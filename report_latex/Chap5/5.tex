
\section{Dataset và Preprocessing}

\textbf{NASA Bearing Dataset:} Dữ liệu vibration từ 4 vòng bi được giám sát đến khi hỏng hoàn toàn. Mỗi sample có 8 channels (cảm biến) với 20480 data points mỗi channel.
\begin{figure}[H]
\centering
\includegraphics[width=0.9\textwidth]{image/bearing.png}
\caption{Thiết bị trong công nghiệp}
\end{figure}

\textbf{Feature extraction:} Từ raw time-series, trích xuất 8 features thống kê:
\begin{itemize}
    \item Mean và Standard Deviation
    \item RMS (Root Mean Square)
    \item Kurtosis và Skewness
    \item Peak-to-Peak
    \item Crest Factor
    \item Form Factor
\end{itemize}

\textbf{Data distribution:} Tổng cộng 32,760 samples được chia thành:

\textit{Balanced (IID):} Mỗi client có đúng 3,276 training samples và 820 test samples.

\textit{Imbalanced (Non-IID):} Phân phối theo power law để mô phỏng thực tế:
\begin{itemize}
    \item Client 0: 9,830 samples (100.0\%)
    \item Client 1: 1,638 samples (14.3\%)
    \item Client 2: 3,932 samples (25.5\%)
    \item Client 3: 3,276 samples (17.5\%)
    \item Client 4: 2,948 samples (13.6\%)
    \item Client 5: 2,620 samples (10.8\%)
    \item Client 6: 2,620 samples (9.8\%)
    \item Client 7: 2,293 samples (7.9\%)
    \item Client 8: 3,276 samples (10.1\%)
    \item Client 9: 329 samples (1.0\%)
\end{itemize}

\begin{table}[H]
\centering
\caption{Phân phối dữ liệu giữa 10 clients}
\begin{tabular}{lrrrr}
\toprule
\textbf{Client} & \textbf{Train (IID)} & \textbf{Train (Non-IID)} & \textbf{Test (IID)} & \textbf{Test (Non-IID)} \\
\midrule
0 & 3,276 & 9,830 (100.0\%) & 820 & 2,458 \\
1 & 3,276 & 1,638 (14.3\%) & 820 & 410 \\
2 & 3,276 & 3,932 (25.5\%) & 820 & 983 \\
3 & 3,276 & 3,276 (17.5\%) & 820 & 820 \\
4 & 3,276 & 2,948 (13.6\%) & 820 & 738 \\
5 & 3,276 & 2,620 (10.8\%) & 820 & 656 \\
6 & 3,276 & 2,620 (9.8\%) & 820 & 656 \\
7 & 3,276 & 2,293 (7.9\%) & 820 & 574 \\
8 & 3,276 & 3,276 (10.1\%) & 820 & 820 \\
9 & 3,276 & 329 (1.0\%) & 820 & 83 \\
\bottomrule
\end{tabular}
\end{table}

\section{Sensor Data Visualization}

\begin{figure}[H]
\centering
\includegraphics[width=0.9\textwidth]{image/sensor_data_visualization.png}
\caption{Visualization của sensor data patterns}
\end{figure}

Biểu đồ cho thấy:
\begin{itemize}
    \item Normal data có pattern khá stable và consistent
    \item Anomaly samples có spikes hoặc deviations rõ ràng từ normal pattern
    \item Các features khác nhau capture các aspects khác nhau của bearing condition
\end{itemize}

\section{Kiến Trúc Mô Hình Autoencoder}

Autoencoder được thiết kế để học representation của dữ liệu normal bearing:

\textbf{Architecture:}
\begin{itemize}
    \item Input layer: 8 features
    \item Encoder: $8 \rightarrow 4 \rightarrow 2$ (bottleneck)
    \item Decoder: $2 \rightarrow 4 \rightarrow 8$
    \item Activation: ReLU
    \item Loss function: Mean Squared Error (MSE)
\end{itemize}

\textbf{Training configuration:}
\begin{itemize}
    \item Optimizer: Adam với learning rate $10^{-3}$
    \item Local epochs: 1 epoch mỗi round
    \item Batch size: 32
    \item Total rounds: 50
\end{itemize}

\textbf{Rationale:} Autoencoder học reconstruct dữ liệu bình thường. Khi gặp anomaly (bearing fault), reconstruction error sẽ cao hơn nhiều, cho phép phát hiện.

\textbf{MSE Formula và Computation:}

Loss function được sử dụng là Mean Squared Error (MSE):
$$MSE = \frac{1}{n} \sum_{i=1}^{n} (x_i - \hat{x}_i)^2$$

Trong đó:
\begin{itemize}
    \item $x_i$: giá trị gốc từ cảm biến thứ $i$
    \item $\hat{x}_i$: giá trị tái tạo từ autoencoder cho cảm biến thứ $i$
    \item $n$: số lượng features (8 sensors trong bearing dataset)
\end{itemize}

\textbf{Decision Rule cho Anomaly Detection:}
$$\text{Classification} = \begin{cases} 
\text{NORMAL} & \text{if } MSE < \text{Threshold} \\
\text{ANOMALY} & \text{if } MSE \geq \text{Threshold}
\end{cases}$$

\textbf{Threshold Calculation:}
$$\text{Threshold} = \text{Percentile}_{95}(\{MSE_i\}_{i=1}^{N})$$

Trong đó $N$ là tổng số test samples trong validation set.

\section{Cấu Hình Decentralized Federated Learning}

Triển khai DFL thuần túy với kiến trúc P2P Ring Topology:

\begin{itemize}
    \item \textbf{Số peers:} 10 (mô phỏng 10 thiết bị IoT)
    \item \textbf{Topology:} Ring - mỗi peer kết nối với 2 peer lân cận
    \item \textbf{Communication:} Peer-to-Peer (P2P) - không có central server
    \item \textbf{Aggregation:} Local tại mỗi peer (trung bình trọng số từ peer trước đó)
    \item \textbf{Total rounds:} 50
    \item \textbf{Local epochs:} 1 epoch mỗi round
    \item \textbf{Batch size:} 128
    \item \textbf{Learning rate:} 0.001
    \item \textbf{Device:} CPU
\end{itemize}

\textbf{Ring Topology Flow:} 
\begin{itemize}
    \item Peer 0 $\rightarrow$ Peer 1 $\rightarrow$ Peer 2 $\rightarrow$ ... $\rightarrow$ Peer 9 $\rightarrow$ Peer 0
    \item Mỗi peer train local model và gửi weights cho peer tiếp theo
    \item Peer nhận weights từ peer trước đó và thực hiện aggregation (averaging)
    \item Không có single point of failure - hoàn toàn phi tập trung
\end{itemize}

\textbf{Ưu điểm của DFL P2P:} 
\begin{itemize}
    \item Tăng tính riêng tư: dữ liệu không bao giờ rời khỏi peer
    \item Giảm bandwidth: chỉ giao tiếp với 2 peer lân cận
    \item Fault tolerance: nếu một peer fail, topology có thể tái cấu hình
    \item Scalable: dễ dàng thêm/bớt peer mà không ảnh hưởng toàn hệ thống
\end{itemize}

\section{Quy Trình Thực Nghiệm}

\textbf{Experiment 1: Balanced (IID)}
\begin{enumerate}
    \item Chia đều 32,760 samples cho 10 peers (3,276 samples/peer)
    \item Khởi tạo model với random weights
    \item Chạy 50 rounds DFL với P2P ring topology
    \item Mỗi peer train local model và trao đổi weights với peer kế tiếp
    \item Đánh giá trên local test set của mỗi peer
\end{enumerate}

\textbf{Experiment 2: Imbalanced (Non-IID)}
\begin{enumerate}
    \item Phân phối không cân bằng theo power law (từ 329 đến 9,830 samples)
    \item Các bước còn lại giống Experiment 1
\end{enumerate}

\textbf{Metrics:}
\begin{itemize}
    \item Training loss: MSE trên training set của mỗi peer
    \item Evaluation loss: MSE trên test set của mỗi peer
    \item Convergence speed: số rounds để đạt loss < 0.005
    \item Loss reduction: chênh lệch giữa initial loss và final loss
    \item Anomaly detection threshold: 95th percentile và Mean+2$\sigma$ của MSE distribution
\end{itemize}
