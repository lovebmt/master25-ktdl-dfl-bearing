
\section{Kết Luận}

Báo cáo này đã trình bày tổng quan về Decentralized Federated Learning (DFL) và tầm quan trọng của nó trong hệ thống IoT. Chúng tôi đã triển khai thành công mô hình DFL với P2P ring topology cho bài toán anomaly detection trên bearing data với 10 peers mô phỏng thiết bị IoT. Kết quả thí nghiệm cho thấy phân phối dữ liệu cân bằng (Balanced) đạt hiệu suất tốt hơn phân phối không cân bằng (Imbalanced) với final eval loss 0.002425 so với 0.002705 (thấp hơn 10.4\%). Hệ thống đạt 100\% accuracy trong anomaly detection với threshold dựa trên 95th percentile.

\textbf{Kết luận chính:} DFL với P2P ring topology là giải pháp khả thi và hiệu quả cho machine learning trên IoT, cung cấp privacy, fault tolerance, và low latency mà centralized approaches không đạt được. Kiến trúc phi tập trung hoàn toàn loại bỏ single point of failure và tăng cường bảo vệ quyền riêng tư, phù hợp cho các ứng dụng industrial IoT đòi hỏi bảo mật cao.

\textbf{Giới hạn của nghiên cứu:}

\begin{itemize}
    \item Nghiên cứu chủ yếu dựa trên mô phỏng phần mềm, chưa triển khai thực tế trên thiết bị IoT với các ràng buộc phần cứng thực tế
    \item Giả định mạng truyền thông ổn định (reliable network), chưa kiểm thử với các điều kiện mất gói tin (packet loss) hoặc độ trễ cao (high latency)
    \item Chưa triển khai các cơ chế bảo vệ chống lại tấn công Byzantine hoặc các mối đe dọa bảo mật khác
    \item Chỉ sử dụng topology tập trung (centralized FL), chưa kiểm thử các topology DFL thuần P2P như ring, mesh, gossip
    \item Bộ dữ liệu chỉ gồm một loại dữ liệu cảm biến (bearing vibration), chưa kiểm thử trên các loại dữ liệu khác
    \item Quy mô thử nghiệm nhỏ (10 clients), chưa kiểm thử khả năng mở rộng với hàng trăm hoặc hàng nghìn node
\end{itemize}

\section{Hướng Phát Triển}

Các hướng phát triển cho nghiên cứu trong tương lai:

\textbf{Cải tiến kỹ thuật:}

\begin{itemize}
    \item \textbf{Alternative topologies:} Thử nghiệm mesh, gossip, star và so sánh với ring topology
    \item \textbf{Security:} Triển khai Byzantine-robust aggregation, differential privacy và secure aggregation protocols
    \item \textbf{Heterogeneity:} Adaptive learning rates, peer selection strategies và asynchronous updates
    \item \textbf{Model compression:} Quantization, gradient compression và knowledge distillation
    \item \textbf{Hardware deployment:} Test trên Raspberry Pi, NVIDIA Jetson hoặc ESP32 với real sensors
    \item \textbf{Scalability:} Mở rộng lên 100-1000 peers với heterogeneous network conditions
\end{itemize}

\textbf{Nghiên cứu nâng cao:}

\begin{itemize}
    \item \textbf{Advanced algorithms:} Personalized DFL, hierarchical architecture, blockchain-integrated DFL
    \item \textbf{Multi-task learning:} Train nhiều tasks đồng thời (anomaly detection, RUL prediction, classification)
    \item \textbf{Adaptive topology:} Dynamic topology thay đổi based on network conditions
    \item \textbf{Continual learning:} Models adapt to concept drift mà không bị catastrophic forgetting
    \item \textbf{Theory:} Convergence guarantees cho DFL với Non-IID data và dynamic topologies
    \item \textbf{Fairness:} Đảm bảo peers với ít data vẫn benefit từ shared knowledge
\end{itemize}

\textbf{Tác động thực tiễn:}

DFL có tiềm năng chuyển đổi các ngành công nghiệp như:
\begin{itemize}
    \item \textbf{Manufacturing:} Giảm downtime nhờ bảo trì dự đoán (predictive maintenance) mà vẫn đảm bảo quyền riêng tư dữ liệu máy móc
    \item \textbf{Healthcare:} Các bệnh viện có thể hợp tác chẩn đoán y khoa mà không cần chia sẻ dữ liệu bệnh nhân, bảo vệ quyền riêng tư và tuân thủ quy định
    \item \textbf{Transportation:} Xe tự hành học chính sách lái xe từ toàn bộ đội xe mà không cần thu thập dữ liệu tập trung
    \item \textbf{Energy:} Lưới điện thông minh (smart grid) tối ưu vận hành nhờ hợp tác giữa các node mà không cần trung tâm
\end{itemize}
Tác động xã hội:
\begin{itemize}
    \item Trao quyền cho cá nhân về quyền sở hữu và kiểm soát dữ liệu cá nhân
    \item Cho phép các tổ chức nhỏ ứng dụng AI mà không cần hạ tầng tập trung lớn
    \item Hỗ trợ tuân thủ các quy định về bảo mật dữ liệu (GDPR, HIPAA)
    \item Thúc đẩy dân chủ hóa AI, giúp AI tiếp cận rộng rãi hơn trong xã hội
\end{itemize}
