
\section{Kết Luận}

Báo cáo này đã trình bày tổng quan về Decentralized Federated Learning (DFL) và tầm quan trọng của nó trong hệ thống IoT. Chúng tôi đã triển khai thành công mô hình DFL với P2P ring topology cho bài toán anomaly detection trên bearing data với 10 peers mô phỏng thiết bị IoT. Kết quả thí nghiệm cho thấy phân phối dữ liệu cân bằng (Balanced) đạt hiệu suất tốt hơn phân phối không cân bằng (Imbalanced) với final eval loss 0.002425 so với 0.002705 (thấp hơn 10.4\%). Hệ thống đạt 100\% accuracy trong anomaly detection với threshold dựa trên 95th percentile.

\textbf{Kết luận chính:} DFL với P2P ring topology là giải pháp khả thi và hiệu quả cho machine learning trên IoT, cung cấp privacy, fault tolerance, và low latency mà centralized approaches không đạt được. Kiến trúc phi tập trung hoàn toàn loại bỏ single point of failure và tăng cường bảo vệ quyền riêng tư, phù hợp cho các ứng dụng industrial IoT đòi hỏi bảo mật cao.

\textbf{Giới hạn của nghiên cứu:}

\begin{itemize}
    \item Mô phỏng phần mềm, chưa deploy trên thiết bị IoT thật với hardware constraints
    \item Giả định reliable network, chưa test với packet loss và high latency
    \item Chưa implement defense mechanisms chống Byzantine attacks
    \item Chỉ test với ring topology, chưa thử nghiệm các topology khác
    \item Quy mô nhỏ (10 peers), chưa test scalability với hàng trăm hoặc hàng nghìn nodes
\end{itemize}

\section{Hướng Phát Triển}

Các hướng phát triển cho nghiên cứu trong tương lai:

\textbf{Cải tiến kỹ thuật:}

\begin{itemize}
    \item \textbf{Alternative topologies:} Thử nghiệm mesh, gossip, star và so sánh với ring topology
    \item \textbf{Security:} Triển khai Byzantine-robust aggregation, differential privacy và secure aggregation protocols
    \item \textbf{Heterogeneity:} Adaptive learning rates, peer selection strategies và asynchronous updates
    \item \textbf{Model compression:} Quantization, gradient compression và knowledge distillation
    \item \textbf{Hardware deployment:} Test trên Raspberry Pi, NVIDIA Jetson hoặc ESP32 với real sensors
    \item \textbf{Scalability:} Mở rộng lên 100-1000 peers với heterogeneous network conditions
\end{itemize}

\textbf{Nghiên cứu nâng cao:}

\begin{itemize}
    \item \textbf{Advanced algorithms:} Personalized DFL, hierarchical architecture, blockchain-integrated DFL
    \item \textbf{Multi-task learning:} Train nhiều tasks đồng thời (anomaly detection, RUL prediction, classification)
    \item \textbf{Adaptive topology:} Dynamic topology thay đổi based on network conditions
    \item \textbf{Continual learning:} Models adapt to concept drift mà không bị catastrophic forgetting
    \item \textbf{Theory:} Convergence guarantees cho DFL với Non-IID data và dynamic topologies
    \item \textbf{Fairness:} Đảm bảo peers với ít data vẫn benefit từ shared knowledge
\end{itemize}

\textbf{Tác động thực tiễn:}

DFL có tiềm năng transform các ngành công nghiệp thông qua predictive maintenance (manufacturing), collaborative medical diagnosis (healthcare), autonomous vehicles learning (transportation), và smart grid optimization (energy), đồng thời empowers individuals với data ownership và privacy, enables AI cho organizations nhỏ, supports regulatory compliance (GDPR, HIPAA), và promotes democratization of AI.
