
\section{Dataset và Phân Phối Dữ Liệu Giữa Các Peers}

Trong kiến trúc Decentralized Federated Learning (DFL), dữ liệu được phân phối giữa 10 peers theo cấu trúc ring topology. Chúng tôi thực hiện hai scenarios phân phối dữ liệu:

\begin{figure}[H]
\centering
\includegraphics[width=0.9\textwidth]{image/data_distribution_visualization.png}
\caption{So sánh phân phối dữ liệu Balanced vs Imbalanced trong DFL}
\end{figure}

\textbf{Phân tích phân phối dữ liệu:}

\begin{table}[H]
\centering
\caption{Thống kê phân phối dữ liệu}
\begin{tabular}{lcccc}
\toprule
\textbf{Distribution} & \textbf{Total Samples} & \textbf{Mean} & \textbf{Std Dev} & \textbf{Range} \\
\midrule
Balanced & 32,760 & 3,276.0 & 0.0 & 3,276 - 3,276 \\
Imbalanced & 32,762 & 3,276.2 & 2,380.9 & 329 - 9,830 \\
\bottomrule
\end{tabular}
\end{table}

\textbf{Quan sát từ visualization:}
\begin{itemize}
    \item \textbf{Balanced Distribution:} Mỗi peer có số lượng samples đồng đều (3,276 samples), tạo ra phân phối cân bằng hoàn hảo với 10\% data cho mỗi peer
    \item \textbf{Imbalanced Distribution:} Peer 0 chiếm 30\% tổng data (9,830 samples), trong khi Peer 9 chỉ có 1\% (329 samples) - chênh lệch gấp 30 lần
    \item Standard deviation của imbalanced distribution (2,380.9) cho thấy độ biến thiên cao
    \item Bar charts minh họa sự chênh lệch đáng kể về số lượng samples giữa các peers trong scenario imbalanced
\end{itemize}

\section{Kiến Trúc Decentralized Federated Learning}

\begin{figure}[H]
\centering
\includegraphics[width=0.9\textwidth]{image/ring_topology.png}
\caption{Sơ đồ kiến trúc Ring Topology trong Decentralized Federated Learning}
\end{figure}

\textbf{Đặc điểm kiến trúc DFL:}
\begin{itemize}
    \item \textbf{Peer-to-Peer Communication:} Không có central server, các peers giao tiếp trực tiếp với nhau theo cấu trúc ring
    \item \textbf{Ring Topology:} Mỗi peer chỉ kết nối với peer kế tiếp, tạo thành một vòng khép kín
    \item \textbf{Local Aggregation:} Mỗi peer tự thực hiện aggregation với model từ peer trước đó
    \item \textbf{Model Architecture:} Autoencoder với cấu trúc Input(8) → Hidden(16) → Latent(4) → Hidden(16) → Output(8)
    \item \textbf{Sequential Update:} Model weights được truyền tuần tự từ Peer 0 → Peer 1 → ... → Peer 9 → Peer 0
\end{itemize}

\textbf{Training Process trong DFL:}
\begin{enumerate}
    \item Mỗi peer train model trên local data
    \item Peer nhận model từ peer trước đó
    \item Thực hiện averaging: $w_{new} = \alpha \cdot w_{local} + (1-\alpha) \cdot w_{received}$
    \item Gửi model đã aggregate đến peer tiếp theo
    \item Sau khi hoàn thành một vòng, bắt đầu round mới
\end{enumerate}

\textbf{Configuration:}
\begin{itemize}
    \item Number of Peers: 10
    \item Training Rounds: 50
    \item Local Epochs: 1
    \item Learning Rate: 0.001
    \item Batch Size: 128
    \item Device: CPU
\end{itemize}

\section{Hiệu Suất Training}

\begin{table}[H]
\centering
\caption{Kết quả thí nghiệm DFL}
\begin{tabular}{lcccc}
\toprule
\textbf{Experiment} & \textbf{Initial Loss} & \textbf{Final Train Loss} & \textbf{Final Eval Loss} & \textbf{Reduction} \\
\midrule
DFL Balanced & 0.041149 & 0.002392 & 0.002425 & 94.19\% \\
DFL Imbalanced & 0.039153 & 0.002842 & 0.002705 & 92.74\% \\
\bottomrule
\end{tabular}
\end{table}

\begin{figure}[H]
\centering
\includegraphics[width=0.9\textwidth]{image/experiments_comparison.png}
\caption{So sánh quá trình training của hai thí nghiệm DFL}
\end{figure}

\textbf{Quan sát từ kết quả:}

\begin{itemize}
    \item \textbf{Balanced Distribution} cho kết quả tốt hơn với final eval loss (0.002425) thấp hơn Imbalanced (0.002705)
    \item Train loss reduction của Balanced (94.19\%) cao hơn đáng kể so với Imbalanced (92.74\%)
    \item Cả hai experiments đều hội tụ ổn định sau khoảng 30-40 rounds
    \item Balanced distribution có learning curve mượt mà hơn, ít fluctuation hơn
\end{itemize}

\textbf{Phân tích chi tiết:}

\begin{enumerate}
    \item \textbf{DFL Balanced:}
    \begin{itemize}
        \item Convergence nhanh và ổn định nhờ phân phối đều dữ liệu
        \item Mỗi peer đóng góp đều nhau vào việc cập nhật model
        \item Giảm bias từ peers có nhiều data
        \item Final loss thấp nhất: 0.002392 (train), 0.002425 (eval)
    \end{itemize}
    
    \item \textbf{DFL Imbalanced:}
    \begin{itemize}
        \item Peer 0 (30\% data) có ảnh hưởng lớn đến model updates
        \item Peers nhỏ (như Peer 9 với 1\% data) có impact hạn chế
        \item Có thể gây overfitting cho data patterns của peers lớn
        \item Final loss cao hơn: 0.002842 (train), 0.002705 (eval)
    \end{itemize}
\end{enumerate}

\textbf{So sánh với Centralized FL:}
\begin{itemize}
    \item DFL không cần central server, giảm single point of failure
    \item Privacy được bảo vệ tốt hơn nhờ P2P communication
    \item Training time có thể dài hơn do sequential updates
    \item Convergence phụ thuộc vào ring topology và data distribution
\end{itemize}

\section{Convergence Analysis}

\begin{figure}[H]
\centering
\includegraphics[width=0.9\textwidth]{image/peer_losses.png}
\caption{Training và Evaluation loss của từng peer trong quá trình training}
\end{figure}

\textbf{Phân tích loss của từng peer:}

\begin{table}[H]
\centering
\caption{Thống kê loss của các peers - Balanced Distribution}
\begin{tabular}{lccccc}
\toprule
\textbf{Peer} & \textbf{Samples} & \textbf{Initial Loss} & \textbf{Final Train} & \textbf{Final Eval} & \textbf{Reduction} \\
\midrule
Peer 0 & 3,276 & 0.04303 & 0.002049 & 0.002438 & 95.24\% \\
Peer 1 & 3,276 & 0.02577 & 0.001977 & 0.001773 & 92.33\% \\
Peer 2 & 3,276 & 0.02174 & 0.002580 & 0.002017 & 88.13\% \\
Peer 3 & 3,276 & 0.05092 & 0.002307 & 0.001638 & 95.47\% \\
Peer 4 & 3,276 & 0.03053 & 0.002087 & 0.002195 & 93.16\% \\
Peer 5 & 3,276 & 0.03723 & 0.002126 & 0.002153 & 94.29\% \\
Peer 6 & 3,276 & 0.05539 & 0.002244 & 0.002087 & 95.95\% \\
Peer 7 & 3,276 & 0.01219 & 0.002105 & 0.002637 & 82.73\% \\
Peer 8 & 3,276 & 0.01435 & 0.001934 & 0.002277 & 86.52\% \\
Peer 9 & 3,276 & 0.04340 & 0.002229 & 0.003102 & 94.86\% \\
\bottomrule
\end{tabular}
\end{table}

\textbf{Quan sát từ balanced distribution:}
\begin{itemize}
    \item Tất cả peers đều hội tụ tốt với final eval loss < 0.0032
    \item Peer 3 có reduction cao nhất (95.47\%), cho thấy khả năng học tốt
    \item Peer 7 có reduction thấp nhất (82.73\%) nhưng vẫn đạt kết quả tốt
    \item Loss curves của các peers tương đối đồng nhất do phân phối data đều
\end{itemize}

\begin{table}[H]
\centering
\caption{Thống kê loss của các peers - Imbalanced Distribution}
\begin{tabular}{lccccc}
\toprule
\textbf{Peer} & \textbf{Samples} & \textbf{Initial Loss} & \textbf{Final Train} & \textbf{Final Eval} & \textbf{Reduction} \\
\midrule
Peer 0 & 9,830 & 0.01663 & 0.002777 & 0.002876 & 83.31\% \\
Peer 1 & 1,638 & 0.04672 & 0.002941 & 0.002362 & 93.70\% \\
Peer 2 & 3,932 & 0.02022 & 0.002707 & 0.002382 & 86.62\% \\
Peer 3 & 3,276 & 0.02027 & 0.003014 & 0.002569 & 85.14\% \\
Peer 4 & 2,948 & 0.02484 & 0.002571 & 0.003229 & 89.65\% \\
Peer 5 & 2,620 & 0.02056 & 0.002962 & 0.002472 & 85.59\% \\
Peer 6 & 2,620 & 0.01199 & 0.002786 & 0.002167 & 76.77\% \\
Peer 7 & 2,293 & 0.02967 & 0.002685 & 0.002279 & 90.95\% \\
Peer 8 & 3,276 & 0.04849 & 0.002990 & 0.003589 & 93.83\% \\
Peer 9 & 329 & 0.04461 & 0.003602 & 0.003129 & 91.93\% \\
\bottomrule
\end{tabular}
\end{table}

\textbf{Quan sát từ imbalanced distribution:}
\begin{itemize}
    \item Peer 0 (30\% data) có initial loss thấp nhất (0.01663) nhưng reduction không cao nhất
    \item Peer 9 (1\% data) có final loss cao nhất (0.003602) do ít data để train
    \item Peers với ít data (Peer 1, 9) có fluctuation nhiều hơn
    \item Average reduction (88.75\%) thấp hơn balanced (91.87\%)
\end{itemize}

\begin{figure}[H]
\centering
\includegraphics[width=0.9\textwidth]{image/final_loss_comparison.png}
\caption{So sánh final loss của các peers trong cả hai scenarios}
\end{figure}

\textbf{Key Insights:}
\begin{itemize}
    \item Balanced distribution cho kết quả đồng nhất hơn giữa các peers
    \item Imbalanced distribution tạo ra sự chênh lệch lớn giữa peers
    \item Peers có nhiều data không nhất thiết có performance tốt nhất
    \item Ring topology giúp knowledge transfer giữa các peers
\end{itemize}

\section{MSE Distribution và Threshold Determination}

\begin{figure}[H]
\centering
\includegraphics[width=0.9\textwidth]{image/mse_distribution_threshold.png}
\caption{Phân phối MSE và xác định ngưỡng anomaly detection cho DFL}
\end{figure}

\textbf{Threshold cho Balanced Distribution:}
\begin{itemize}
    \item 95th Percentile Threshold: 0.006297
    \item Mean MSE: 0.003014
    \item Std Dev: 0.001641
    \item Median: 0.002873
    \item Mean + 2$\sigma$: 0.006296
\end{itemize}

\textbf{Threshold cho Imbalanced Distribution:}
\begin{itemize}
    \item 95th Percentile Threshold: 0.007214
    \item Mean MSE: 0.003858
    \item Std Dev: 0.001678
    \item Median: 0.003652
    \item Mean + 2$\sigma$: 0.007214
\end{itemize}

\textbf{Histogram Analysis:}
\begin{itemize}
    \item Balanced distribution có MSE tập trung hơn, phản ánh sự đồng nhất trong training
    \item Imbalanced distribution có spread rộng hơn, cho thấy sự khác biệt giữa các peers
    \item 95th percentile được chọn để minimize false positives trong anomaly detection
    \item Threshold của imbalanced cao hơn 25.4\% so với balanced
\end{itemize}

\textbf{Cumulative Distribution Function (CDF):}
\begin{itemize}
    \item 95\% samples có MSE dưới threshold trong cả hai scenarios
    \item CDF curve của balanced steeper hơn, cho thấy consistency tốt hơn
    \item Imbalanced có long tail, phản ánh outliers từ peers có ít data
\end{itemize}

\textbf{Box Plot Analysis:}
\begin{itemize}
    \item Balanced: IQR compact hơn, ít outliers
    \item Imbalanced: IQR rộng hơn, nhiều outliers hơn
    \item Median của imbalanced cao hơn 27.1\% so với balanced
    \item Outliers được xác định rõ ràng bởi threshold line
\end{itemize}

\section{Anomaly Detection Performance}

\subsection{Balanced Distribution Results}

Mô hình được test trên 4 scenarios với threshold = 0.006297:

\begin{table}[H]
\centering
\caption{Kết quả anomaly detection - Balanced Distribution}
\begin{tabular}{lccc}
\toprule
\textbf{Test Case} & \textbf{MSE Error} & \textbf{Threshold} & \textbf{Result} \\
\midrule
Normal Sample & 0.002002 & < 0.006297 & \textcolor{green}{NORMAL} \\
Scenario 1: Sensor Error & 0.212897 & > 0.006297 & \textcolor{red}{ANOMALY} \\
Scenario 2: High Vibration & 0.028053 & > 0.006297 & \textcolor{red}{ANOMALY} \\
Scenario 3: Negative Values & 0.007802 & > 0.006297 & \textcolor{red}{ANOMALY} \\
\bottomrule
\end{tabular}
\end{table}

\textbf{Phân tích Balanced:}
\begin{itemize}
    \item Normal sample có error (0.002002) chỉ bằng 31.8\% threshold
    \item Sensor error có error cao nhất (0.212897), gấp 33.8 lần threshold
    \item High vibration có error (0.028053) gấp 4.5 lần threshold
    \item Negative values có error (0.007802) gấp 1.2 lần threshold
    \item \textbf{Accuracy: 100\%} - Phân biệt hoàn toàn giữa normal và anomalies
\end{itemize}

\subsection{Imbalanced Distribution Results}

Mô hình được test trên 4 scenarios với threshold = 0.007214:

\begin{table}[H]
\centering
\caption{Kết quả anomaly detection - Imbalanced Distribution}
\begin{tabular}{lccc}
\toprule
\textbf{Test Case} & \textbf{MSE Error} & \textbf{Threshold} & \textbf{Result} \\
\midrule
Normal Sample & 0.001684 & < 0.007214 & \textcolor{green}{NORMAL} \\
Scenario 1: Sensor Error & 0.188280 & > 0.007214 & \textcolor{red}{ANOMALY} \\
Scenario 2: High Vibration & 0.040921 & > 0.007214 & \textcolor{red}{ANOMALY} \\
Scenario 3: Negative Values & 0.010645 & > 0.007214 & \textcolor{red}{ANOMALY} \\
\bottomrule
\end{tabular}
\end{table}

\textbf{Phân tích Imbalanced:}
\begin{itemize}
    \item Normal sample có error (0.001684) chỉ bằng 23.3\% threshold
    \item Sensor error có error cao nhất (0.188280), gấp 26.1 lần threshold
    \item High vibration có error (0.040921) gấp 5.7 lần threshold
    \item Negative values có error (0.010645) gấp 1.5 lần threshold
    \item \textbf{Accuracy: 100\%} - Phân biệt hoàn toàn giữa normal và anomalies
\end{itemize}

\begin{figure}[H]
\centering
\includegraphics[width=0.9\textwidth]{image/anomaly_detection_comparison.png}
\caption{So sánh anomaly detection giữa Balanced và Imbalanced}
\end{figure}

\textbf{So sánh giữa hai distributions:}

\begin{table}[H]
\centering
\caption{So sánh detection performance}
\begin{tabular}{lcccc}
\toprule
\textbf{Test Case} & \textbf{Balanced Error} & \textbf{Imbalanced Error} & \textbf{Difference} & \textbf{Detection} \\
\midrule
Normal & 0.002002 & 0.001684 & -15.9\% & Both OK \\
Sensor Error & 0.212897 & 0.188280 & -11.6\% & Both OK \\
High Vibration & 0.028053 & 0.040921 & +45.9\% & Both OK \\
Negative Values & 0.007802 & 0.010645 & +36.4\% & Both OK \\
\bottomrule
\end{tabular}
\end{table}

\textbf{Kết luận về Anomaly Detection:}
\begin{itemize}
    \item \textbf{Accuracy:} Cả hai models đạt 100\% accuracy trong việc phân biệt anomalies
    \item \textbf{Sensitivity:} Imbalanced model có errors cao hơn cho cả normal và anomaly samples
    \item \textbf{Robustness:} Balanced model có reconstruction errors ổn định hơn
    \item \textbf{Threshold Selection:} 95th percentile phù hợp cho cả hai scenarios
    \item \textbf{Practical Use:} Balanced distribution được recommend cho anomaly detection tasks
\end{itemize}

\section{Tổng Kết và Đánh Giá}

\subsection{So sánh Performance giữa Balanced và Imbalanced}

\begin{table}[H]
\centering
\caption{Bảng tổng hợp kết quả thực nghiệm DFL}
\begin{tabular}{lcc}
\toprule
\textbf{Metrics} & \textbf{Balanced} & \textbf{Imbalanced} \\
\midrule
\textbf{Training Performance} & & \\
Initial Train Loss & 0.041149 & 0.039153 \\
Final Train Loss & 0.002392 & 0.002842 \\
Train Loss Reduction & 94.19\% & 92.74\% \\
\midrule
\textbf{Evaluation Performance} & & \\
Initial Eval Loss & 0.030544 & 0.029577 \\
Final Eval Loss & 0.002425 & 0.002705 \\
Eval Loss Reduction & 92.07\% & 90.86\% \\
\midrule
\textbf{Anomaly Detection} & & \\
Threshold (95th percentile) & 0.006297 & 0.007214 \\
Normal Sample Error & 0.002002 & 0.001684 \\
Anomaly Detection Accuracy & 100\% & 100\% \\
\midrule
\textbf{Data Distribution} & & \\
Mean Samples per Peer & 3,276.0 & 3,276.2 \\
Std Dev & 0.0 & 2,380.9 \\
Min-Max Range & 3,276-3,276 & 329-9,830 \\
\bottomrule
\end{tabular}
\end{table}

\subsection{Key Findings}

\textbf{1. Impact của Data Distribution:}
\begin{itemize}
    \item Balanced distribution cho performance tốt hơn về mọi mặt
    \item Final eval loss thấp hơn 10.4\% (0.002425 vs 0.002705)
    \item Learning curve mượt mà và ổn định hơn
    \item Convergence nhanh hơn và đạt local minimum tốt hơn
\end{itemize}

\textbf{2. Decentralized FL Architecture:}
\begin{itemize}
    \item Ring topology hoạt động hiệu quả với 10 peers
    \item Sequential model passing đảm bảo knowledge transfer
    \item Không cần central server, tăng privacy và robustness
    \item P2P communication giảm communication bottleneck
\end{itemize}

\textbf{3. Anomaly Detection Capability:}
\begin{itemize}
    \item Cả hai models đạt 100\% accuracy trong detection
    \item 95th percentile threshold phù hợp cho production use
    \item Balanced model có reconstruction errors ổn định hơn
    \item Robust với nhiều loại anomalies khác nhau
\end{itemize}

\textbf{4. Scalability và Practical Considerations:}
\begin{itemize}
    \item System scale tốt với 10 peers, có thể mở rộng thêm
    \item Training time: ~50 rounds để converge hoàn toàn
    \item Memory efficient: mỗi peer chỉ cần store local data
    \item Network overhead thấp: chỉ truyền model weights giữa adjacent peers
\end{itemize}

\subsection{Recommendations}

\textbf{Cho Production Deployment:}
\begin{enumerate}
    \item \textbf{Data Distribution:} Ưu tiên balanced distribution khi có thể
    \item \textbf{Threshold Selection:} Sử dụng 95th percentile với periodic recalibration
    \item \textbf{Monitoring:} Track individual peer performance để phát hiện stragglers
    \item \textbf{Model Updates:} Implement checkpoint saving sau mỗi round
    \item \textbf{Fault Tolerance:} Xử lý peer failures với timeout mechanisms
\end{enumerate}

\textbf{Cho Future Work:}
\begin{enumerate}
    \item Test với topology khác (mesh, hierarchical)
    \item Implement dynamic peer weighting dựa trên data quality
    \item Explore adaptive learning rates per peer
    \item Add encryption cho model weights transmission
    \item Benchmark với centralized FL và standalone models
\end{enumerate}

\subsection{Limitations}

\begin{itemize}
    \item Sequential updates trong ring topology có thể chậm với nhiều peers
    \item Peer failures có thể break the ring và require recovery mechanism
    \item Không có global view của training progress
    \item Imbalanced data vẫn ảnh hưởng đáng kể đến performance
    \item Threshold cần được tune cho từng application domain
\end{itemize}

\subsection{Conclusion}

Thực nghiệm cho thấy Decentralized Federated Learning với ring topology là một giải pháp khả thi cho anomaly detection trong IoT bearing monitoring systems. Balanced data distribution mang lại performance tốt nhất với final eval loss 0.002425 và 100\% anomaly detection accuracy. Kiến trúc DFL loại bỏ single point of failure và tăng cường privacy bảo vệ, phù hợp cho các ứng dụng industrial IoT đòi hỏi bảo mật cao.
